\documentclass{exam}

% Language and font encodings
\usepackage[greek,english]{babel}
\usepackage[utf8x]{inputenc}

% Math
\usepackage{amsmath} % math
\usepackage{amssymb} % symbols
\usepackage{amsthm}

% Multiple choice
\usepackage{multicol}

% Page numbering
\footer{}{\thepage}{}

\renewcommand{\qedsymbol}{$\blacksquare$}

\title{\textgreek{Μια συλλογή σύντομων ασκήσεων στα \\ Μαθηματικά Γ' λυκείου}}
\date{}
\author{\textgreek{Παναγιώτης Πετρίδης}}

\begin{document}
    \maketitle
    \thispagestyle{empty}
    \begin{questions}
        % Exercise #1
        \newpage
        \setcounter{page}{1}
        \setcounter{equation}{0}
        
        \question \textgreek{Αν συνάρτηση $f:\mathbb{R}\rightarrow\mathbb{R}$ 
        παραγωγίσημη με $f(\alpha) = \gamma$ και $f(\beta) = \delta$. Να αποδείξετε ότι: }

        \begin{equation*}
            \int_{\alpha}^{\beta}f(x)dx + \int_{\gamma}^{\delta}f^{-1}(x)dx = \beta\cdot\delta - \alpha\cdot\gamma
        \end{equation*}

        % Exercise #2
        \setcounter{equation}{0}
        \vspace{0.5cm}

        \question \textgreek{Αν συνάρτηση $f:[-1,1]\rightarrow\mathbb{R}$ 
        με τύπο $f(x) = \sqrt{1 - x^2}$ να υπολογίσετε το ολοκλήρωμα: }

        \begin{equation*}
            E(\Omega) = \int_{0}^{1}f(x)dx
        \end{equation*}

        % Exercise #3
        \setcounter{equation}{0}
        \vspace{0.5cm}
        
        \question \textgreek{α) Ποιά απο τις παρακάτω σχέσεις ισχύει?}

        \begin{multicols}{4}
            $i$) $e^\pi > \pi^e$
            \\
            $ii$) $e^\pi < \pi^e$
            \\
            $iii$) $e^\pi = \pi^e$
        \end{multicols}

        \textgreek{β) Να αποδείξετε την απάντησή σας.}

        % Exercise #4
        \setcounter{equation}{0}
        \vspace{0.5cm}
        
        \question \textgreek{Να υπολογίσετε, αν υπάρχουν, τα όρια: }
        \begin{align*}
            \lim_{\kappa\to0}\int_{-1}^{1}(1-x)^\kappa dx   &&   \lim_{\kappa\to+\infty}\int_{-1}^{1}(1-x)^\kappa dx
        \end{align*}

        % Exercise #5
        \setcounter{equation}{0}
        \vspace{0.5cm}
        
        \question \textgreek{Να υπολογίσετε το εμβαδό του χωρίου που περικλίετε απο τον $x'x$ και τις ευθείες της εξίσωσης:}
        \begin{equation*}
            y^2 - 2y -x^2 + 1 = 0
        \end{equation*}
        
        % Exercise #6
        \setcounter{equation}{0}
        \vspace{0.5cm}

        \question \textgreek{Εαν γνωρίζετε ότι $e = \lim_{x\to0}(1+x)^\frac{1}{x}$ να αποδείξετε ότι: }
        \begin{equation*}
            \lim_{h\to0}\frac{ln(x+h) - lnx}{h} = \frac{1}{x}
        \end{equation*}
        \textgreek{Σημείωση: μην χρησημοποιήσετε τον τύπο $(lnx)' = \frac{1}{x}$}

        % Exercise #7
        \setcounter{equation}{0}
        \vspace{0.5cm}

        \question \textgreek{Αν $f:x\in[0,\frac{\pi}{2})\cup(\frac{\pi}{2}, \frac{3\pi}{2})\rightarrow\mathbb{R}$ 
        η συνάρτηση $f(x) = \varepsilon\varphi x$ να υπολογίσετε το εμβαδό του χωρίου $\Omega$ 
        που περικλείεται απο την $C_f$, τον $x'x$ και την ευθεία $y=\frac{2\sqrt{3}}{7\pi}x$.}

        % Exercise #8
        \setcounter{equation}{0}
        \vspace{0.5cm}

        \question \textgreek{Έστω συνάρτηση $f(x) = \eta\mu kx$ με $x,k>0$ και 
        $x\leq\frac{\pi}{k}$. Να αποδείξετε ότι το εμβαδό του χωρίου $\Omega$ που 
        περικλίετε απο την $f$, την ευθεία $y=1$ και τον άξονα $y'y$
        τίνει στο 0 καθως το $k$ τίνει στο $+\infty$}

    \end{questions}
\end{document}